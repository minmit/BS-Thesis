
% A B S T R A C T

\pagestyle{empty}

\begin{center}
\مهم{چکیده}
\end{center}

با گسترش شبکه‌های کامپیوتری و به ویژه اینترنت، محدودیت استفاده از IP به عنوان لایه‌ی سوم در شبکه برای پخش گسترده‌ی اطلاعات بیش از پیش خود را نمایان می سازد. شبکه‌های مبتنی بر داده‌ی نام‌گذاری‌شده (NDN) به تازگی به عنوان راهکاری برای حل این مسئله مطرح شده اند. در این شبکه‌ها به جای این که به گره‌های موجود در شبکه شناسه داده شود، برای داده‌ها شناسه در نظر گرفته می شود. امروزه این شبکه‌ها به دلیل تفاوت بنیادین با شبکه‌های کنونی مورد توجه بسیاری از محققین قرار گرفته اند. با این حال ایده‌های مقدماتی مطرح شده برای مسیریابی در این شبکه ها با مشکلاتی روبه‌رو هستند. 

در این رساله به ارائه‌ی پروتکلی جدید برای مسیریابی در شبکه های ‌NDN می‌پردازیم. این پروتکل مبتنی بر بردار فاصله می باشد و نسبت به پروتکل‌های قبلی از تعداد پیغام‌های کمتری برای مسیریابی استفاده می‌کند. هم‌چنین پیدا کردن مسیرهای متفاوت به یک داده، برای پشتیبانی از ویژگی چند-مسیری شبکه‌های NDN، با سربار محاسباتی کمتری نسبت به پروتکل‌های قبلی انجام می‌گیرد. لذا به دلیل ویژگی های خاص آن، این روش می تواند جایگزین مناسبی برای روش های پیشین مطرح شده در این حوزه باشد.
\صفحه‌جدید
