
% A B S T R A C T

\pagestyle{empty}

\begin{center}
\مهم{چکیده}
\end{center}

در این رساله به بررسی \کج{مسئله‌ی راه فرار مستطیل‌ها} می‌پردازیم که در مسیر‌دهی بر روی برد‌های دیجیتال مورد نیاز است. مسئله‌ی راه فرار مستطیل‌ها را می‌توان این‌گونه تعریف کرد:
مجموعه‌ی $S$ شامل $n$ مستطیل داده شده که همگی بر روی یک ناحیه‌ی مستطیلی بزرگ به نام \کج{قاب} قرار گرفته‌اند، به شکلی که اضلاع مستطیل‌ها همگی موازی مرز‌های قاب هستند. هدف، فراری دادن هر کدام از این مستطیل‌های عضو $S$ به یکی از چهار جهت اصلی است، به گونه‌ای که بیشینه چگالی نقاط قاب، کمینه شود.
برای هر نقطه‌ی $p$ از قاب، چگالی نقطه‌ی $p$، تعداد مستطیل‌هایی است که در فرارشان از آن نقطه عبور می‌کنند.
 
در این رساله، ابتدا سختی این مسئله را بررسی می‌کنیم و نشان می‌دهیم که با فرض $P \neq NP$، الگوریتمی چند‌جمله‌ای با ضریب تقریب بهتر از ${3} \over {2}$
برای این مسئله وجود ندارد.
سپس برای حالتی که بیشینه چگالی مجاز برابر با یک است، یک الگوریتم چند‌جمله‌ای با زمان اجرای $O(n ^ 4)$ ارائه می‌کنیم که در مقایسه با الگوریتم پیشین این مسئله با زمان اجرا $O(n ^ 6)$، از زمان اجرای بهتری برخوردار است.
در پایان، به ازای هر مقدار $\epsilon > 0$، با این شرط که مقدار جواب به اندازه‌ی کافی بزرگ باشد، یک الگوریتم تقریبی احتمالی با ضریب تقریب $1 + \epsilon$ ارائه می‌دهیم. این درحالی است که ضریب تقریب بهترین الگوریتم تقریبی قبلی $4$ است.

\صفحه‌جدید
