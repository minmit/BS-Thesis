
% A B S T R A C T

\pagestyle{empty}

\begin{latin}

\EnglishAbstractFont

\begin{center}
\textbf{abstract}
\end{center}

The rapid growth of computer networks, specifically Internet, in recent years has introduced a new challenge in using the Internet Protocol (IP) as the thin waist of network architecture. IP, initially designed for providing ubiquitious interconnectivity, has faced serious complications as means of extensive content distribution. Thus, Named Data Networks (NDN) have been proposed to solve these problems by using data name instead of its address when requesting for data. These networks have attracted significant scientific attention for their fundamental paradigm shift form IP-based networks. For NDN to work efficiently over networks, a dynamic routing protocol is needed to find and rank various paths to data produced in the network.

In this paper, we propose a distance-vector routing protocol for NDN networks. Our proposed protocol reduces the number of routing messages needed to update routing paths. Besides, the protocol imposes less computational overhead for calculating multipaths to the same destination than its current counterparts. These features make this protocol an suitable alternative for existing routing protocols.
\end{latin}

\newpage
