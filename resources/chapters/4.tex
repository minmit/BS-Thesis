
\فصل{مسیریابی}
در هر شبکه‌ای، این مسیریاب‌ها هستند که بسته‌ها را به مقصد خود می‌رسانند. هر مسیریاب به تعدادی مسیریاب دیگر متصل است و مجموعه آن‌ها یک گراف را تشکل می‌دهند. هر مسیریاب باید بداند که بسته‌ای که به آن رسیده است را روی کدام واسط خود بفرستند. 
پروتکل مسیریابی مشخص می‌کند که مسیریاب‌ها چگونه با یکدیگر در ارتباط‌‌ هستند و اطلاعاتی را رد و بدل می‌کنند تا در نهایت بتوانند مسیری که یک بسته باید طی کند را مشخص کنند و بسته را در مسیر درست هدایت کنند. هر مسیریاب در ابتدا فقط اطلاعات پیوندهایی را دارد که به طور مستقیم به آنها متصل است و این وظیفه پروتکل مسیریابی است که یه نوعی اطلاعات لازم را به هر مسیریاب برساند تا او بتواند به ازای هربسته، مسیر درست را تشخیص دهد. 

پروتکل‌های مسیریابی به دو دسته اصلی تقسیم می‌شوند:‌

\شروع{فقرات}
\فقره پروتکل‌های وضعیت پیوند\زیرنوشت{link-state protocols}، مانند OSPF\زیرنوشت{Open Shortest Path First}
و
 IS-IS
 \زیرنوشت{Intermediate System to Intermediate System}
\فقره پروتکل‌های بردار فاصله\زیرنوشت{distance-vector protocols} مانند RIP\زیرنوشت{Routing Information Protocol}
و IGRP\زیرنوشت{Interior Gateway Routing Protocol}
\پایان{فقرات}
 در ادامه به توضیح مختصری راجع به این دو مدل مسیریابی می‌پردازیم:
 
 \قسمت{پروتکل وضعیت پیوند}
 ایده اصلی پشت این پروتکل این است که هر مسیریاب یک نقشه کلی از گراف شبکه را در اختیار دارد و با توجه به آن، خودش کوتاهترین مسیر را محاسبه می‌کند. 
 
 \قسمت{پروتکل بردار فاصله}

