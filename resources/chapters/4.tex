
\فصل{الگوریتم دقیق برای چگالی واحد}

در این فصل، یک الگوریتم با زمان اجرای $O(n ^ 4)$ برای \decision{1} ارائه می‌کنیم. به بیان دیگر، یک نمونه از مسئله‌ی راه فرار مستطیل‌ها داده شده‌است که در آن مکان اولیه‌ی هیچ دو مستطیلی هم‌پوشانی ندارند و هدف فراری دادن مستطیل‌هاست با این شرط که چگالی هیچ نقطه‌ای از قاب بیش‌تر از یک نشود. همان‌گونه که پیش‌تر گفته‌شد، زمان اجرای بهترین الگوریتم قبلی برای این مسئله، الگوریتمی با زمان $O(n ^ 6)$ بوده که در \cite{BoundaryRec} ارائه شده‌است. در این بخش، با کمک برنامه‌ریزی پویا\زیرنوشت{Dynamic Programming}، الگوریتمی با زمان اجرای $O(n ^ 4)$ برای این مسئله به دست خواهیم‌آورد. به این منظور، ابتدا این مسئله‌ی بهینه‌سازی را که آن را بیشینه فرار مجزا\زیرنوشت{Maximum Disjoint Escaping} می‌نامیم، در نظر بگیرید:

\شروع{مسئله}[بیشینه فرار مجزا]
\label{prob:Maximum-Disjoint-Escaping}

یک نمونه از مسئله‌ی راه فرار مستطیل‌ها داده شده‌است که در آن مکان هیچ دو مستطیلی هم‌پوشانی ندارند. بیش‌ترین تعداد مستطیلی را بیابید که با چگالی یک می‌توانند فرار کنند.

\پایان{مسئله}

لازم به ذکر است که در مسئله‌ی~\ref{prob:Maximum-Disjoint-Escaping}، مکان اولیه‌ی مستطیل‌های فرار نکرده هم اهمیت دارد: یک مستطیل در مسیر فرار خود نمی‌تواند با مکان اولیه‌ی هیچ مستطیل دیگری - حتی مستطیل‌های فرار نکرده - برخورد کند.

مستطیل‌های $R_1, \ldots, R_n$ را در نظر بگیرید به گونه‌ای که اضلاع این مستطیل‌ها موازی محور‌های مختصات هستند و هیچ دو مستطیلی هم‌پوشانی ندارند. بدیهی است که قاب را می‌توان هر ناحیه‌ی مستطیل دلخواهی با مرز‌های موازی محور‌ها در نظر گرفت که همه‌ی این $n$ مستطیل را در بر بگیرد. فرض کنید که این $n$ مستطیل بر حسب ضلع پایینی خود مرتب شده‌اند، به گونه‌ای که ضلع پایینی $R_i$ از ضلع پایینی $R_{i + 1}$ بالا‌تر نیست $(1 \leq i < n)$.

برای مستطیل $R_i$ و  $d \in \{left, right, up, down\}$، جهت $d$ را برای مستطیل $R_i$
\کج{آزاد} می‌نامیم، اگر $R_i$ در صورت فرار در جهت $d$ از روی مکان اولیه‌ی هیچ مستطیلی عبور نکند. بنا بر تعریف، آزاد بودن یک جهت برای یک مستطیل وابسته به جهت فرار هیچ مستطیلی نیست. شکل~\ref{fig:free-directions} را ببینید. برای هر مستطیل $R_i$، جهت‌های آزاد این مستطیل را می‌توان به سادگی در زمان $O(n)$ به دست آورد، پس با یک پیش‌پردازش\زیرنوشت{Preprocess} در زمان $O(n ^ 2)$ جهت‌های آزاد همه‌ی مستطیل‌های داده‌شده را می‌توان یافت. گذشته از این، مجموعه‌ی $\{v_1, \ldots, v_k\}$ را مجموعه‌ی همه‌ی خط‌های عمودی در نظر می‌گیریم که ضلع‌های عمودی مستطیل‌ها بر روی آن‌ها قرار گرفته‌اند. فرض کنید این خط‌ها از چپ به راست مرتب شده‌اند. این خط‌ها را نیز در پیش‌پردازش با همان زمان $O(n ^ 2)$ می‌توان به دست آورد.\زیرنویس{همه‌ی این پیش‌پردازش‌ها را در زمان $O(n \log n)$ هم می‌توان انجام داد.}

\شکل‌پی‌دی‌اف{6}{جهت‌های آزاد برای هر یک از مستطیل‌ها در \decision{1}}{free-directions}

برای حل مسئله‌ی \ref{prob:Maximum-Disjoint-Escaping}، ابتدا دو زیر مسئله‌ی ساده‌ی زیر را در نظر می‌گیریم:

\شروع{فقرات}

\فقره $One \MyDash Direction(i, l, r)$

بیش‌ترین تعداد مستطیل از میان مستطیل‌های $R_{1}, \ldots, R_{i}$ که بین دو خط عمودی $v_{l}$ و $v_{r}$ قرار گرفته‌اند و می‌توانند به سمت بالا فرار کنند.

\فقره $Two \MyDash Direction(i, l, r)$

بیش‌ترین تعداد مستطیل از میان مستطیل‌های $R_{1}, \ldots, R_{i}$ که بین دو خط عمودی $v_{l}$ و $v_{r}$ قرار گرفته‌اند و می‌توانند به سمت بالا یا پایین فرار کنند.

\پایان{فقرات}

این دو زیر‌مسئله به سادگی با الگوریتم‌های حریصانه\زیرنوشت{Greedy} قابل حل هستند. برای یافتن مقدار $One \MyDash Direction(i, l, r)$ کافی است از میان مستطیل‌های $R_{1}, \ldots, R_{i}$ که بین دو خط  $v_{l}$ و $v_{r}$ قرار گرفته‌اند، تعداد مستطیل‌هایی را بیابیم که جهت بالا برای آن‌ها آزاد است. هم‌چنین برای $Two \MyDash Direction(i, l, r)$ باید تعداد مستطیل‌هایی در بین مستطیل‌های گفته‌شده را یافت که برای آن‌ها جهت بالا یا پایین، آزاد باشد. توجه کنید که اگر دو مستطیل در جهت عمودی (بالا یا پایین) فرار کنند به گونه‌ای که جهت فرارشان آزاد باشد، آن‌گاه مسیر فرارشان برخورد نخواهد داشت.

بنا به آن‌چه گفته شد، مقادیر $One \MyDash Direction$ و $Two \MyDash Direction$ را می‌توان در زمان $O(n ^ 4)$ برای همه‌ی سه‌تایی‌های $(i, l, r)$به دست آورد\زیرنویس{این مقادیر با کمک برنامه‌ریزی پویا در زمان $O(n ^ 3)$ نیز قابل محاسبه هستند.} و در آرایه‌هایی ذخیره نمود.

اکنون زیر‌مسئله‌های زیر را در نظر بگیرید:

\شروع{مسئله}[فرار به سه جهت]
\label{prob:3-Direction}

زیر‌مسئله‌ی $No \MyDash Left \MyDash Escape(i, b, l, r)$ بیش‌ترین تعداد مستطیل در میان مستطیل‌های $R_{1}, \ldots, R_{i}$ تعریف می‌شود که با شرایط زیر می‌توانند فرار کنند:

\شروع{فقرات}

\فقره هیچ مستطیلی نمی‌تواند به سمت چپ فرار کند.

\فقره فقط مستطیل‌هایی که سمت راست خط عمودی $v_{b}$ قرار دارند، می‌توانند فرار کنند.

\فقره فقط مستطیل‌هایی که بین خط‌های $v_{l}$ و $v_{r}$ قرار دارند، می‌توانند به سمت پایین فرار کنند.

\پایان{فقرات}

شکل~\ref{fig:dynamic} را ببینید.

\شکل‌پی‌دی‌اف{8}{یک نمونه از مسئله‌ی فرار به سه جهت}{dynamic}

زیر‌مسئله‌ی $No \MyDash Right \MyDash Escape(i, b, l, r)$ نیز به طریق مشابه قابل تعریف است، با این تفاوت که در $No \MyDash Right \MyDash Escape$، هیچ مستطیلی به سمت راست نمی‌تواند فرار کند.

\پایان{مسئله}

برای یافتن مقدار $No \MyDash Left \MyDash Escape(i, b, l, r)$ به صورت بازگشتی، می‌توان همه‌ی گزینه‌های ممکن برای $R_{i}$ (که پایین‌ترین مستطیل در بین مستطیل‌های مورد نظر است) را در نظر گرفت. نخستین گزینه این است که $R_{i}$ فرار نکند. در این صورت، بیش‌ترین تعداد مستطیلی که با رعایت محدودیت‌های مسئله می‌توانند فرار کنند، برابر است با $No \MyDash Left \MyDash Escape(i - 1, b, l, r)$ که به صورت بازگشتی قابل محاسبه است. سایر گزینه‌ها، فرار $R_{i}$ به یکی از سه جهت پایین، بالا و راست هستند که به صورت جداگانه در زیر بررسی شده‌اند. در همه‌ی حالت‌ها فرض بر این است که جهت مورد بررسی برای $R_{i}$ آزاد است و فرار در آن جهت با همه‌ی محدودیت‌های مسئله سازگاری دارد.

در ضمن فرض کنید خطوط $v_{\alpha}$ و $v_{\beta}$ خط‌هایی هستند که ضلع‌های عمودی $R_{i}$ روی آن‌ها قرار گرفته‌اند و $\alpha < \beta$ (به بیان دیگر، ضلع سمت چپ روی $v_{\alpha}$ و ضلع سمت راست روی $v_{\beta}$ قرار گرفته‌است.)

\شروع{فقرات}

\فقره فرار به سمت \کج{پایین}

اگر $R_{i}$ به سمت پایین فرار کند، هیچ محدودیت جدیدی برای فرار مستطیل‌های $R_{1}, \ldots, R_{i - 1}$ ایجاد نخواهد شد. بنابراین، بیش‌ترین تعداد مستطیل از بین این $i - 1$ مستطیل که می‌توانند با رعایت محدودیت‌های مسئله فرار کنند، برابر است با $No \MyDash Left \MyDash Escape(i - 1, b, l, r)$ که به صورت بازگشتی قابل محاسبه است.

\فقره فرار به سمت \کج{بالا}

با فرار $R_{i}$ به سمت بالا، یک محدودیت بر فرار $i - 1$ مستطیل دیگر افزوده می‌شود: مستطیل‌هایی که در سمت راست $v_{\beta}$ قرار ندارند، نمی‌توانند به سمت راست فرار کنند، چراکه در این صورت، با مسیر فرار $R_{i}$ برخورد خواهند کرد. با توجه به این نکته، مستطیل‌هایی را در که بین $v_{b}$ و $v_{\beta}$ قرار دارند، در نظر بگیرید. بسته به جایگاه قرارگیری $v_l$ و $v_r$، بیش‌ترین تعداد مستطیل از بین این مستطیل‌ها که می‌توانند فرار کنند، با استفاده از زیر مسئله‌های $One \MyDash Direction$ و $Two \MyDash Direction$ به دست می‌آید. از سوی دیگر، بیش‌ترین تعداد مستطیل از $R_{1}, \ldots, R_{i - 1}$ که سمت راست $v_{\beta}$ قرار دارند و می‌توانند فرار کنند، برابر است با
$$No \MyDash Left \MyDash Escape(i - 1, max(b, \beta), l, r)$$
که به صورت بازگشتی قابل محاسبه است.

\فقره فرار به سمت \کج{راست}

اگر $R_{i}$ به سمت راست فرار کند، از بین $i - 1$ مستطیل دیگر، مستطیل‌هایی که سمت چپ $v_{\alpha}$ قرار ندارند، نمی‌توانند به پایین فرار کنند. پس اگر مستطیلی بخواهد به پایین فرار کند، باید نه تنها سمت چپ $v_{r}$ که سمت چپ $v_{\alpha}$ نیز قرار داشته‌باشد. به این ترتیب، بیش‌ترین تعداد مستطیل از میان این $i - 1$ مستطیل که می‌توانند فرار کنند، برابر است با:
$$No \MyDash Left \MyDash Escape(i - 1, b, l, min\set{r, \alpha})$$

\پایان{فقرات}

زیر‌مسئله‌ی $No \MyDash Left \MyDash Escape$ مطابق آن‌چه گفته شد، با در نظر گرفتن همه‌ی گزینه‌های ممکن برای $R_{i}$ به صورت بازگشتی قابل حل است. همین الگوریتم بازگشتی را می‌توان به الگوریتمی مبتنی بر برنامه‌ریزی پویا تبدیل کرد. این ترتیب مقدار $No \MyDash Left \MyDash Escape$ برای همه‌ی چهارتایی‌هایی $(i, b, l, r)$ در زمان $O(n ^ 4)$ به دست می‌آید. هم‌چنین برای زیر‌مسئله‌ی $No \MyDash Right \MyDash Escape$ نیز با الگوریتم مشابهی می‌توان پاسخ را به ازای همه‌ی چهارتایی‌های $(i, b, l, r)$ به دست آورد. با داشتن این مقادیر، مسئله‌ی کلی زیر را حل خواهیم کرد:

\شروع{مسئله}
\label{prob:Max-Routing-Dynamic}

به ازای اعداد $0 \leq i \leq n$ و $1 \leq l, r \leq k$، بیش‌ترین تعداد مستطیل از بین مستطیل‌های $R_{1}, \ldots, R_{i}$ را بیابید که با این محدودیت می‌توانند فرار کنند: تنها مستطیل‌هایی که سمت راست $v_{l}$ و سمت چپ $v_{r}$ قرار دارند، می‌توانند به سمت پایین فرار کنند.

\پایان{مسئله}

\begin{alg}{$Max \MyDash Route(i, l, r)$}{Max-Route}
		
	\If {$i = 0$}
		\State Return $0$
	\EndIf

	\State $ans_{n} \leftarrow Max \MyDash Route(i - 1, l, r)$

	\State $ans_{d} \leftarrow 0, ans_{u} \leftarrow 0, ans_{l} \leftarrow 0, ans_{r} \leftarrow 0$

	\State $\alpha, \beta \leftarrow$ indices of the vertical lines through the left and the right sides of $R_{i}$
	
	\If {\emph{down} is feasible for \emph{$R_i$}}
		\State $ans_{d} \leftarrow Max \MyDash Route(i - 1, l, r) + 1$ 
	\EndIf

	\If {\emph{left} is feasible for \emph{$R_i$}}
		\State $ans_{l} \leftarrow Max \MyDash Route(i - 1, \max\!\set{l, \beta}, r) + 1$ 
	\EndIf

	\If {\emph{right} is feasible for \emph{$R_i$}}
		\State $ans_{r} \leftarrow Max \MyDash Route(i - 1, l, \min\!\set{r, \alpha}) + 1$ 
	\EndIf

	\If {\emph{up} is feasible for \emph{$R_i$}}
		\State $ans_{u} \leftarrow No \MyDash Right \MyDash Escape(i - 1, \alpha, l, r) + No \MyDash Left \MyDash Escape  (i - 1, \beta, l, r)  + 1$
	\EndIf

	\State Return $\max\!\set{ans_{n}, ans_{d}, ans_{u}, ans_{l}, ans_{r}}$

\end{alg}

این مسئله که حالت کلی‌تری از مسئله‌ی \ref{prob:Maximum-Disjoint-Escaping} است، مشابه دو زیر‌مسئله‌ی گفته‌شده با در نظر گرفتن همه‌ی گزینه‌های ممکن برای $R_{i}$ و به صورت بازگشتی قابل حل است. الگوریتم~\ref{algo:Max-Route} را ببینید.

الگوریتم بازگشتی~\ref{algo:Max-Route} را نیز می‌توان با بهره‌گیری از برنامه‌ریزی پویا بازنویسی کرد و به این ترتیب، مقدار $Max \MyDash Route$ برای همه‌ی سه‌تایی‌های $(i, l, r)$ در زمان $O(n ^ 3)$ به دست می‌آید.

در پایان، برای یافتن پاسخ مسئله‌ی~\ref{prob:Maximum-Disjoint-Escaping} تنها داشتن مقدار $Max \MyDash Route (n, 1, k)$ کافی است که در آن، $1$ اندیس چپ‌ترین و $k$ اندیس راست‌ترین خط عمودی است. بدیهی است که اگر این مقدار برابر $n$ باشد، به معنی آن است که همه‌ی این $n$ مستطیل می‌توانند در چگالی $1$ فرار کنند. اکنون قضیه‌ی زیر را می‌توان نتیجه گرفت:

\شروع{قضیه}

به ازای $k = 1\ $، مسئله‌ی~\ref{prob:k-REP} در زمان $O(n ^ 4)$ قابل حل است که در آن، $n$ تعداد مستطیل‌های ورودی است.

\پایان{قضیه}

\شروع{اثبات}

کافی است ابتدا بررسی شود که آیا $n$ مستطیل داده‌شده هم‌پوشانی دارند یا نه. اگر دو مستطیل هم‌پوشانی داشته‌باشند، بدیهی است که مستطیل‌ها نمی‌توانند به گونه‌ای فرار کنند که چگالی همه‌ی نقاط قاب حد‌اکثر $1$ باشد، چراکه برخی ار نقاط قاب را بیش از یک مستطیل پوشانده‌اند. در غیر این صورت، کافی است پاسخ مسئله‌ی~\ref{prob:Maximum-Disjoint-Escaping} با عدد $n$ مقایسه شود.

\پایان{اثبات}
