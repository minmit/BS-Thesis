

\chapter{سختی مسمسنیتبمسینتبمنیبت}
در این بخش، ابتدا سختی \decision{2} را بررسی می‌کنیم و نشان می‌دهیم که این مسئله در کلاس پیچیدگی \کامل{} قرار دارد. این نتیجه را برای حالت خاص‌تری از مسئله که در آن هیچ دو مستطیلی از $n$ مستطیل ورودی هم‌پوشانی ندارند، ثابت می‌کنیم. سپس نتیجه می‌گیریم با فرض $NP \neq P$، برای مسئله‌ی~\ref{prob:REP} الگوریتمی تقریبی با ضریب تقریب بهتر از ${3} \over {2}$ نمی‌توان یافت.

\قسمت{سختی مسئله برای چگالی 2}

همان گونه که اشاره شد، سختی مسئله‌ی~\ref{prob:k-REP} برای چگالی $3$ پیش‌تر در \cite{REP} بررسی شده‌است. در این جا، سختی این مسئله را به ازای $k = 2$ نشان خواهیم داد. پیش از بررسی سختی مسئله‌ی~\ref{prob:k-REP}، ابتدا مسئله‌ی مشهور \SAT{3}\زیرنوشت{$3$-SAT} را معرفی می‌کنیم:

\شروع{مسئله}[\SAT{3}]
\label{prob:3-SAT}

برای مجموعه‌ی $X = \set{x_{1}, \ldots, x_{n}}$ از متغیر‌هایی با دامنه‌ی $\set{0, 1}$، $m$ عبارت\زیرنوشت{Clause} به شکل $C_{i} = l_{i, 1} \vee l_{i, 2} \vee l_{i, 3}$ داده‌شده‌اند که در آن، هر $l_{i, \lambda}$ یکی از $n$ متغیر‌ عضو $X$ یا نقیض یکی آن‌هاست.
آیا می‌توان متغیر‌های مجموعه‌ی $X$ را به گونه‌ای مقداردهی کرد که مقدار همه‌ی این $m$ عبارت‌ منطقی، برابر $1$ شود؟

\پایان{مسئله}

مسئله‌ی~\ref{prob:3-SAT} یکی از مشهور‌ترین مسئله‌های \کامل{} شناخته‌شده است~\cite{Karp}. ما در این بخش با فرض \کامل{} بودن مسئله‌ی~\ref{prob:3-SAT}، قضیه‌ی زیر را ثابت می‌کنیم:

\شروع{قضیه}
\label{theorem:NP-complete-2REP}

برای $k = 2$، مسئله‌ی~\ref{prob:k-REP} حتی با این محدودیت که هیچ دو مستطیلی هم‌پوشانی ندارند، یک مسئله‌ی \کامل{} است.

\پایان{قضیه}

\شروع{اثبات}

این که مسئله‌ی مورد نظر در کلاس پیچیدگی \ان‌پی{} قرار می‌گیرد، بدیهی است. پس برای نشان دادن \کامل{} بودن این مسئله برای $k = 2$، کافی است \سخت{} بودن آن را ثابت کنیم. این کار را با کاهش\زیرنوشت{Reduction} از مسئله‌ی~\ref{prob:3-SAT} انجام می‌دهیم. به بیان ساده‌تر، ثابت می‌کنیم که اگر \decision{2} در حالتی که هم‌پوشانی وجود ندارد، الگوریتمی با زمان اجرای چند‌جمله‌ای داشته‌باشد، آن‌گاه مسئله‌ی~\ref{prob:3-SAT} نیز در زمان چند‌جمله‌ای قابل حل است. به این منظور، برای یک نمونه از مسئله‌ی \SAT{3}، یک نمونه از مسئله‌ی راه فرار مستطیل‌ها به شکل زیر می‌سازیم:

یک قاب مستطیلی در نظر می‌گیریم و آن را به صورت مجازی به چهار ناحیه تقسیم می‌کنیم. این ناحیه‌ها را مطابق شکل~\ref{fig:lb}، ناحیه‌ی \کج{بالا}\زیرنوشت{Top Region}، ناحیه‌ی \کج{چپ}\زیرنوشت{Left Region}، ناحیه‌ی \کج{متغیر‌ها}\زیرنوشت{Variable Region} و ناحیه‌ی \کج{عبارت‌ها}\زیرنوشت{Clause Region} می‌نامیم.

\شکل‌پی‌دی‌اف{12}{کاهش از مسئله‌ی \SAT{3}}{lb}

\شروع{فقرات}

\فقره به ازای هر متغیر $x_{j}$، دو مستطیل با نام‌های $v_{j}$ و $\bar{v_{j}}$ متناظر با $x_{j}$ و $\bar{x_{j}}$ همان گونه که در شکل~\ref{fig:lb} نشان داده‌شده، در ناحیه‌ی متغیر‌ها به صورت افقی کنار هم قرار می‌دهیم. این مستطیل‌های متناظر با متغیرها باید به گونه‌ای قرار بگیرند که یک خط عمودی یا افقی، دو مستطیل مربوط به دو متغیر مختلف را قطع نکند. فرض کنید این مستطیل‌ها را مستطیل‌های \کج{نوع یک} بنامیم. علاوه بر مستطیل‌های نوع یک، یک مستطیل بلند نیز به صورت عمودی در سمت راست ناحیه‌ی متغیر‌ها قرار می‌گیرد. شکل~\ref{fig:lb}  را ببینید.

\فقره برای هر عبارت $C_i = l_{i, 1} \vee l_{i, 2} \vee l_{i, 3}$، سه مستطیل در بخش عبارت‌ها قرار می‌دهیم. این سه مستطیل روی یک خط افقی به گونه‌ای قرار می‌گیرند که زیر مستطیل‌های نوع یک متناظر با $l_{i, i}$، $l_{i, 2}$ و $l_{i, 3}$ باشند. همه‌ی این مستطیل‌ها را نیز مستطیل‌های \کج{نوع دو} می‌نامیم. همان گونه که در مثال شکل~\ref{fig:lb}  نشان داده‌شده، یک خط عموی یا افقی نباید هیچ دو مستطیلی را که وابسته به دو عبارت مختلف هستند، قطع نماید.

علاوه بر این مستطیل‌های نوع دو، دو مستطیل بلند به صورت عمودی در سمت چپ و سمت راست این ناحیه و هم‌چنین یک مستطیل بلند افقی در پایین آن قرار می‌دهیم.

\فقره برای هر متغیر، در سمت چپ مستطیل‌های نوع یک متناظر با آن، $5$ مربع کوچک که شکلی صلیبی ساخته‌اند، در ناحیه‌ی چپ قرار می‌گیرند. این شکل‌های صلیبی را \کج{سد} می‌نامیم. دقت کنید که $5$ مربع یک سد به هر روشی که فرار کنند، چگالی نقطه‌ای روی یکی از آن‌ها حد‌اقل $2$ خواهد شد. سد‌های ناحیه‌ی چپ به گونه‌ای قرار می‌گیرند که با یک خط افقی یا عمودی نتوان دو سد مختلف را قطع کرد.

\فقره بالای هر مستطیل نوع دو، یک سد در ناحیه‌ی بالا قرار می‌گیرد. سپس اگر بالای یک مستطیل نوع یک هیج سدی قرار نگرفت، برای آن مستطیل نیز یک سد در ناحیه‌ی بالا قرار می‌دهیم. سد‌های ناحیه‌ی بالا را نیز به شکلی قرار می‌دهیم که یک خط عمودی یا افقی نتواند دو سد مختلف را قطع کند.

\پایان{فقرات}

اکنون فرض کنید در نمونه‌ی ساخته‌شده، مستطیل‌ها بتوانند به گونه‌ای فرار کنند که چگالی هیچ نقطه‌ای از قاب بیش‌تر از $2$ نشود. می‌خواهیم ثابت کنیم یک مقداردهی برای متغیر‌های $\set{x_{1}, \ldots, x_{n}}$ وجود دارد که به ازای آن مقداردهی، همه‌ی عبارت‌ها مقدار $1$ می‌گیرند.

با توجه به مکان قرارگیری سد‌ها در ناحیه‌ی چپ و ناحیه‌ی بالا، هیچ مستطیل نوع یکی نمی‌تواند به چپ یا بالا فرار کرده‌باشد، پس هر کدام از مستطیل‌های نوع یک به راست یا پایین فرار کرده‌اند. از طرفی، با توجه به قرار‌گیری مستطیل بلندی در سمت راست ناحیه‌ی متغیر‌ها، دو مستطیل نوع یکی که متناظر با متغیر یکسانی هستند، نمی‌توانند هر دو به راست فرار کرده‌باشند؛ چراکه در این صورت چگالی نقطه‌ای روی مستطیل بلند سمت راست ناحیه‌ی متغیر‌ها، بیش‌تر از $2$ می‌شد. برای هر متغیر $x_{i}$، اگر $v_{i}$ به سمت راست فرار کرده‌بود، مقدار $x_{i}$ را برابر $1$ قرار می‌دهیم و اگر $\bar{v_{i}}$ به سمت راست فرار کرده‌بود، مقدار $\bar{x_{i}}$ را برابر با $1$ در نظر می‌گیریم. اگر هم هیچ یک از این دو مستطیل به سمت راست فرار نکرده‌بودند (در واقع فرار هر دو به سمت پایین بود)، مقدار $x_{i}$ را $1$ و در نتیجه مقدار $\bar{x_{i}}$ را $0$ می‌گیریم. ادعا می‌کنیم که با این مقداردهی، همه‌ی عبارت‌ها برابر $1$ خواهند شد.

به خاطر سد‌های ناحیه‌ی بالا، هیچ یک از مستطیل‌های نوع دو نمی‌تواند به سمت بالا فرار کند. هم‌چنین با توجه به قرارگیری دو مستطیل بلند سمت چپ و راست ناحیه‌ی عبارت‌ها، برای هر عبارت $C_i = l_{i, 1} \vee l_{i, 2} \vee l_{i, 3}$، حداکثر یکی از سه مستطیل نوع دو وابسته به این عبارت می‌تواند به چپ فرار کند و حد‌اکثر هم یکی به راست. بنابراین، حد‌اقل یکی از این سه مستطیل نوع دو به سمت پایین فرار کرده‌است. فرض کنید که مستطیل متناظر با $l_{i, \lambda}$ به پایین فرار کرده‌باشد ($\lambda \in \set{1, 2, 3}$). مستطیل نوع یکی که بالای این مستطیل قرار گرفته‌است، باید به سمت راست فرار کرده‌باشد، چون اگر به پایین فرار کرده‌باشد، چگالی نقطه‌ای روی مستطیل بلند پایین ناحیه‌ی عبارت‌ها بیش از $2$ خواهد شد. پس با توجه به نحوه‌ی مقداردهی متغیر‌ها، مقدار $l_{i, \lambda}$ و در نتیجه مقدار عبارت $C_i$ برابر $1$ است. به این ترتیب ثابت شد که با مقداردهی ارائه‌شده، مقدار همه‌ی عبارت‌‌ها برابر $1$ خواهد شد.

در سوی دیگر، باید ثابت کنیم که اگر برای نمونه‌ی داده‌شده از مسئله‌ی \SAT{3}، یک مقداردهی وجود داشته‌باشد که همه‌ی عبارت‌ها را $1$ کند، آن‌گاه مستطیل‌ها می‌توانند به گونه‌ای فرار کنند که بیشینه چگالی نقاط قاب $2$ باشد. به این منظور، برای هر متغیر $x_{i}$، اگر $x_{i} = 1$، آن‌گاه جهت فرار $v_{i}$ را سمت راست و جهت فرار $\bar{v_{i}}$ را پایین در نظر در نظر می‌گیریم. در غیر این صورت، $v_{i}$ را به سمت پایین و $\bar{v_{i}}$ را به سمت راست فراری می‌دهیم.

برای هر عبارت $C_i = l_{i, 1} \vee l_{i, 2} \vee l_{i, 3}$، حد‌اقل یک $\lambda \in \set{1, 2, 3}$ وجود دارد که مقدار $l_{i, \lambda}$ برابر $1$ باشد. برای فرار مستطیل نوع دو متناظر با $l_{i, \lambda}$ جهت پایین در نظر گرفته می‌شود و از بین دو مستطیل نوع دو دیگری که وابسته به همین عبارت هستند، مستطیل سمت چپ به سمت چپ و مستطیل دیگر به راست فرار می‌کند.

هم‌چنین مستطیل بلند سمت راست ناحیه‌ی متغیر‌ها به بالا، مستطیل‌های بلند سمت چپ و سمت راست ناحیه‌ی عبارت‌ها هر دو به پایین و مستطیلی که در پایین این ناحیه قرار گرفته‌است به سمت راست فراری داده می‌شود. در مورد سد‌ها هم کافی است از هر سد، مربع‌های وسطی و پایینی به سمت چپ فرار کنند و سه مستطیل دیگر به سمت بالا. به این ترتیب، همه‌ی مستطیل‌ها می‌توانند به گونه‌ای فرار کنند که بیشینه چگالی نقاط قاب برابر $2$ باشد. گفتنی است که با این شیوه‌ی فرار، چگالی هیچ نقطه‌ای روی اضلاع قاب بیش‌تر از $1$ نخواهد شد.

بنابر آن‌چه گفته‌شد، در صورت حل \decision{2} در زمان چند‌جمله‌ای، مسئله‌ی \SAT{3} نیز در زمان چند‌جمله‌ای قابل حل خواهد بود. پس مسئله‌ی~\ref{prob:k-REP} برای $k = 2$ در کلاس پیچیدگی \کامل{} قرار دارد.

\پایان{اثبات}

\قسمت{سختی مسئله برای چگالی بیش‌تر از 2}

پیش‌تر، سختی \decision{2} بررسی شد. در این‌جا، نتیجه‌ی مشابهی را با استفاده از استقرا به ازای هر عدد طبیعی ثابت $k \geq 2$ به دست می‌آوریم.

\شروع{قضیه}
\label{theorem:NP-completeness-kREP}
برای هر عدد طبیعی $k \geq 2$، مسئله‌ی~\ref{prob:k-REP} حتی برای حالتی که هیچ دو مستطیلی هم‌پوشانی نداشته‌باشند، در کلاس \کامل{} قرار دارد.

\پایان{قضیه}

\شروع{اثبات}

پایه‌ی استقرا معادل قضیه‌ی~\ref{theorem:NP-complete-2REP} است که ثابت شد. اکنون فرض کنید که متناظر با نمونه‌ی داده‌شده‌ای از مسئله‌ی~\ref{prob:3-SAT}، $I_{\Delta - 1}$ نمونه‌ای از \decision{\Delta - 1} باشد ($\Delta > 2$). برای ساختن $I_{\Delta}$ همان گونه که در شکل~\ref{fig:lb-rec} نشان داده‌شده، یک مربع بزرگ $Q$ در وسط می‌گذاریم و چهار نمونه از $I_{\Delta - 1}$  را در اطراف آن به گونه‌ای قرار می‌دهیم که یک خط عمودی یا افقی نتواند قاب هیچ دوتایی از آن‌ها را قطع کند.

\شکل‌پی‌دی‌اف{8}{ساختن $I_{\Delta}$ با استفاده از چهار نمونه از $I_{\Delta - 1}$ برای مسئله‌ی~\ref{prob:k-REP}}{lb-rec}

ادعا می‌کنیم که
پاسخ \decision{\Delta} برای نمونه‌ی $I_{\Delta}$ 
\بلی{} است اگر و تنها اگر جواب همین مسئله به ازای $k = \Delta - 1$ برای نمونه‌ی $I_{\Delta - 1}$
\بلی{} باشد.
%در $I_{\Delta}$ مستطیل‌ها می‌توانند با بیشینه چگالی $\Delta$ فرار کنند، اگر و تنها اگر مستطیل‌ها در $I_{\Delta - 1}$ بتوانند به گونه‌ای فرار کنند که بیشنه چگالی $\Delta - 1$ شود.
ابتدا فرض کنید مستطیل‌های $I_{\Delta}$ به شکلی فرار کرده‌اند که چگالی هیچ نقطه‌ای از قاب بیش‌تر از $\Delta$ نیست. با توجه به این که $Q$ به یکی از چهار جهت فرار کرده‌است و در نتیجه از روی یکی از چهار نمونه‌ی $I_{\Delta - 1}$ عبور کرده‌است، می‌توان نتیجه گرفت که اگر $I_{\Delta - 1}$ به تنهایی در نظر گرفته‌شود، مستطیل‌ها می‌توانند به گونه‌ای فرار کنند که بیشینه چگالی، $\Delta - 1$ شود. در دیگر سوی، باید ثابت کنیم که اگر مستطیل‌ها در $I_{\Delta - 1}$ بتوانند با بیشینه چگالی $\Delta - 1$ فرار کنند، آن‌گاه در $I_{\Delta}$، مستطیل‌ها می‌توانند به گونه‌ای فرار کنند که بیشینه چگالی $\Delta$ شود. دقت کنید که مستطیل‌های چهار نمونه‌ی $I_{\Delta - 1}$ در $I_{\Delta}$ باید به گونه‌ای فرار کنند که چگالی هیچ نقطه‌ای روی مکان اولیه‌ی $Q$ از $\Delta$ بیش‌تر نشود. مشاهده‌ی زیر اثبات قضیه‌ی~\ref{theorem:NP-completeness-kREP} را کامل می‌کند.

\شروع{مشاهده}

به ازای هر $\Delta \geq 2$، اگر مستطیل‌‌های $I_{\Delta}$ بتوانند به گونه‌ای فرار کنند که چگالی هیچ نقطه‌ای بیش از $\Delta$ نشود، آن‌گاه این مستطیل‌ها را می‌توان به گونه‌ای فراری داد که

\شروع{فقرات}

\فقره چگالی نقاط ضلع بالایی قاب کم‌تر از $\Delta$ (حد‌اکثر $\Delta - 1$) باشد.

\فقره چگالی هر نقطه‌ای روی سه ضلع دیگر قاب حداکثر $1$ باشد.

\پایان{فقرات}

\پایان{مشاهده}

در زیربخش قبلی، این نتیجه برای $\Delta = 2$ به دست آمده‌بود. از طرفی دیگر، فرض کنید مستطیل‌های هر چهار $I_{\Delta - 1}$ در $I_{\Delta}$ به گونه‌ای فرار کرده‌باشند که برای هر یک از این چهار نمونه، بیشینه چگالی روی ضلع بالایی قاب $\Delta - 2$ باشد و چگالی نقاط سایر اضلاع قاب هم حد‌اکثر $1$.
به سادگی می‌توان دید که چگالی هیچ نقطه‌ای از مکان اولیه‌ی $Q$ بیش از $\Delta$ نیست. هم‌چنین اگر خود $Q$ به سمت بالا فرار کند، آن‌گاه محدودیت‌های گفته‌شده در مشاهده برای چگالی نقاط روی قاب $I_{\Delta}$ رعایت شده‌است.

\پایان{اثبات}

\قسمت{تقریب‌ناپذیری}

با استفاده از قضیه‌ی~\ref{theorem:NP-complete-2REP}، می‌توان قضیه‌ی زیر را ثابت کرد:

\شروع{قضیه}[تقریب‌نا‌پذیری]

برای مسئله‌ی~\ref{prob:REP} الگوریتمی تقریبی با زمان چند‌جمله‌ای و ضریب تقریب بهتر از ${3} \over {2}$ نمی‌توان یافت مگر آن‌که $NP = P$.

\پایان{قضیه}

\شروع{اثبات}

گفتنی است که این نتیجه‌ی تقریب‌ناپذیری را نیز برای حالت بدون هم‌پوشانی می‌توان ثابت کرد. اگر الگوریتمی با ضریب تقریب کم‌تر از $\alpha < {{3} \over {2}}$ وجود داشته‌باشد، آن‌گاه به ازای یک ورودی از مسئله‌ی راه فرار مستطیل‌ها، در زمان چند‌جمله‌ای می‌توان دریافت که آیا مستطیل‌ها می‌توانند به گونه‌ای فرار کنند که چگالی هیج نقطه‌ای از قاب بیش از $2$ نشود: مستطیل‌ها را به عنوان ورودی به الگوریتم تقریبی دارای ضریب تقریب $\alpha$ می‌دهیم و پاسخ به دست آمده را با عدد $3$ مقایسه می‌نماییم.

\شروع{فقرات}

\فقره اگر پاسخ مسئله‌ی~\ref{prob:REP} حداکثر $2$ باشد، آن‌گاه الگوریتم تقریبی باید پاسخی کم‌تر $2 \times {{3} \over {2}} = 3$ بیابد.

\فقره اگر پاسخ مسئله‌ی~\ref{prob:REP} حداقل $3$ باشد، پاسخی که الگوریتم تقریبی مورد نظر به دست می‌آورد نیز حد‌اقل $3$ خواهد بود.

\پایان{فقرات}

\پایان{اثبات}
