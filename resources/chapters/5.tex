
\فصل{کارهای پیشین}

همان‌طور که در فصل‌های قبلی اشاره شد، شبکه‌های مبتنی بر داده‌ی نام‌گذاری شده به تازگی به عنوان معماری جدید برای اینترنت مطرح شده‌اند و از این رو کارهای محدودی در زمینه‌ی مسیریابی پویا در آن‌ها انجام گرفته است. در این بخش به بررسی کارهای پیشین قابل توجهی که در این زمینه انجام شده است می‌پردازیم.

از اولین پروتکل‌های مسیریابی که برای شبکه‌های NDN پیشنهاد شد، در مقاله‌ی Dai و بقیه \ref{two-layer} بود. در این پروتکل از یک معماری دو لایه استفاده می‌شود. لایه‌ی زیرین که لایه‌ی نگهدار توپولوژی شبکه \زیرنوشت{Topology Maintaining Layer} (TM) است، با پروتکلی مشابه OSPF توپولوژی شبکه را به دست آورده و به روز نگه می‌دارد و کوتاه‌ترین مسیرها را به دست می‌آورد. در این لایه از بسته‌های درخواست و داده‌ی NDN استفاده نمی‌شود بلکه مشابه شبکه‌های مبتنی بر IP به هر مسیریاب یک شناسه نسبت می‌دهد و هر مسیریاب در صورت تغییر وضعیت یکی از پیوند‌‌های مجاور خود به همسایگانش یک بسته‌ی خبری وضعیت پیوند \زیرنوشت{Link State Advertisement} (LSA) می‌فرستد تا تغییر وضعیت را به آن‌ها اطلاع دهد. در واقع در صورتی که از این پروتکل روی یک شبکه‌ی مبتنی بر IP استفاده شود، شناسه‌ی مسیریاب‌ها همان آدرس IP آن‌ها خواهد بود. لایه‌ی بالایی لایه‌ای است که برای توزیع محتوا مورد استفاده قرار می‌گیرد و لایه‌ی اعلام پیشوند \زیرنوشت{Prefix Announcing Layer}  )PA) نام دارد. در این لایه پیشوند‌های نام داده‌ها در دو حالت فعال و غیرفعال را توزیع می‌کند تا پایگاه اطلاعات ارسال (FIB) با استفاده از آن‌ها ساخته شود. در این لایه نیز از بسته‌های درخواست و داده‌ی NDN استفاده نمی‌شود. این امر سبب می‌شود که از خصوصیات ذاتی NDN مانند امضای دیجیتالی بسته‌ها بهره ببرند.

از دیگر کارهایی که در زمینه‌ی مسیریابی در شبکه‌های NDN انجام شده است، ارائه‌ی پروتکل OSPFN توسط Zhang و بقیه \ref{ospfn} است. این پروتکل در واقع گسترش‌یافته‌ی پروتکل OSPF برای شبکه‌های مبتنی بر IP است. در این پروتکل یک بسته‌ی خبری وضعیت پیوند جدید تعریف می‌کند تا پیشوند‌ها را در پیغام‌های مسیریابی حمل کنند. هم‌چنین بهترین مسیر به هر پیشوند در FIB ثبت می‌شود و اپراتورهای شبکه می‌توانند به صورت دستی چند مسیر جایگزین برای مسیریاب تعریف کنند تا در صورت نیاز از آن‌ها استفاده کند. با وجود این‌که OSPFN در نهایت می‌تواند FIB را به صورت پویا با پیشوندها پر کند، با محدودیت‌های جدی روبه‌رو است. این پروتکل نیز مانند پروتکل قبلی از آدرس IP به عنوان شناسه‌ی مسیریاب‌ها استفاده می‌کند و در نتیجه بهره‌ای از بسته‌های درخواست و داده‌ی NDN نمی‌برد. هم‌چنین تنها اولین بهترین مسیر را برای هر پیشوند در نظر می‌گیرد و استفاده از قابلیت چند-مسیری ارسال بسته‌ها در NDN تنها با مسیرهایی وارد شده توسط اپراتور شبکه امکان‌پذیر است و در واقع از ارسال چند-مسیری پشتیبانی نمی‌کند.

مقاله‌ی Torres و بقیه \ref{controller} یک پروتکل مسیریابی مبتنی بر کنترل‌کننده پیشنهاد می‌دهد. در این پروتکل برای پیشگیری از زیاد شدن بیش از اندازه‌ی حجم FIB عنصر جدیدی به نام کنترل‌کننده به شبکه اضافه می‌کند که مسئولیت نگهداری مکان داده‌ها و مسیریابی را به عهده دارند. در مرحله‌ی راه‌اندازی این پروتکل کنترل‌کننده‌ها توپولوژی را به دست می‌آورند و مسیرهای بین مسیریاب‌ها را محاسبه می‌کنند. این کنترل‌کننده‌ها علاوه بر کوتاه‌ترین مسیرها، مسیر‌های جایگزین با هزینه‌ی بیشتر را نیز برای کنترل ترافیک شبکه محاسبه خواهند کرد. پس از راه‌اندازی، داده‌های نام‌گذاری شده و مکان آن‌ها در کنترل‌کننده‌ها ثبت می‌شوند و در نتیجه کنترل‌کننده‌ها می‌توانند مسیرهای موردنیاز را در مسیریاب‌ها کارگذاری کنند. هم‌چنین مسیریاب‌ها می‌توانند برای مسیریابی داده‌های جدید به کنترل‌کننده‌ها درخواست دهند. هم‌چنین برای جلوگیری از انفجار داده‌های مربوط به مکان داده‌ها در کنترل‌کننده‌ها از جدول توزیع‌شده‌ی درهم‌سازی \زیرنوشت{Distributed Hash Table} (DHT) استفاده می‌شود. با وجود این که این پروتکل روی NDN و با بسته‌های درخواست و داده‌ی آن کار می‌کند، به دلیل استفاده از بسته‌های درخواست با قالب خاصی برای جستجوی کنترل‌کننده‌ها، ممکن است با سربار زیادی مواجه شود. 

قابل‌توجه‌ترین و جدیدترین کاری که در زمینه‌ی مسیریابی در شبکه‌های NDN انجام شده است، کاری است که Zhang و دوستان \ref{nlsr} انجام داده‌اند. در این کار یک پروتکل مبتنی بر وضعیت پیوند \زیرنوشت{Link State}به نام NLSR ارائه می‌شود که روی NDN اجرا می‌شود، به این معنی که از بسته‌های درخواست و داده‌ی آن برای مسیریابی استفاده می‌کند. همانند OSPF که یک پروتکل مبتنی بر وضعیت است، در این پروتکل نیز از بسته‌های خبری وضعیت پیوند برای گسترش وضعیت پیوندها در سرتاسر شبکه استفاده می‌‌شود. هر مسیریاب توپولوژی شبکه را از روی این بسته‌ها برای خود می‌سازد و مسیرها را از روی آن محاسبه می‌کند. از جمله ویژگی‌های مثبت این پروتکل پشتیبانی از چند-مسیری است. در حقیقت هر مسیریاب مسیرهای منتهی به یک پیشوند را از تمامی واسط‌ها محاسبه کرده و آن‌ها را رتبه‌بندی می‌کند و در FIB قرار می‌دهد تا ارسال بر اساس آن‌ها انجام بگیرد. هم‌چنین در NLSR دیگر از آدرس IP مسیریاب‌ها به عنوان شناسه استفاده نمی‌شود 

