
\فصل{الگوریتم تقریبی}

در این فصل، دو الگوریتم تقریبی برای مسئله‌ی راه فرار مستطیل‌ها را بررسی خواهیم‌کرد. از آن‌جایی که الگوریتم‌های تقریبی این فصل مبتنی بر برنامه‌ریزی صحیح و برنامه‌ریزی خطی است، ابتدا یک برنامه‌ریزی صحیح برای مسئله ارائه می‌کنیم.

پیش از ارائه‌ی برنامه‌ریزی صحیح معادل با مسئله‌ی~\ref{prob:REP}، ابتدا فرض کنید همه‌ی اضلاع مستطیل‌ها را از دو طرف گسترش دهیم تا اضلاع قاب را قطع کنند. به این ترتیب، قاب به شکل مشبک در خواهد آمد. شکل~\ref{fig:grid} را ببینید. به سادگی می‌توان دید که مستقل از چگونگی فرار مستطیل‌ها، چگالی همه‌ی نقاط درون یک \کج{سلول} از این شبکه با هم برابر است، پس به جای چگالی نقاط قاب، می‌توان چگالی را به یک سلول نسبت داد. از سوی دیگر، بدیهی است که تعداد سلول‌ها از $O(n ^ 2)$ خواهد بود. 

\شکل‌پی‌دی‌اف{8}{سلول‌ها برای یک نمونه از مسئله‌ی راه فرار مستطیل‌ها}{grid}

اکنون می‌توان یک برنامه‌ریزی صحیح نوشت که معادل مسئله‌ی~\ref{prob:REP}  باشد. مستطیل‌های ورودی را $R_{1}, \ldots, R_{n}$ در نظر بگیرید. به ازای هر $1 \leq i \leq n$، چهار متغیر $x_{i, l}$، $x_{i, r}$، $x_{i, u}$ و $x_{i, d}$ با دامنه‌ی $\set{0, 1}$ تعریف می‌کنیم. منظور از $1$ بودن مقدار $x_{i, l}$، فرار کردن $R_{i}$ به سمت چپ است. به طریق مشابه، $1$ بودن مقدار $x_{i, r}$، $x_{i, u}$ و $x_{i, d}$ را به ترتیب معادل فرار $R_{i}$ به راست، بالا و پایین در نظر می‌گیریم. فرض کنید این متغیر‌ها را \کج{متغیرهای جهت} نام‌گذاری کنیم.

برای هر سلول $c$، مجموعه‌ای به نام $P_{c}$ به این شکل تعریف می‌کنیم: به ازای هر $1 \leq i \leq n$ و هر جهت $\lambda \in \set{l, r, u, d}$، دوتایی $(i, \lambda)$ در $P_{c}$ قرار دارد، اگر و تنها اگر مستطیل $R_{i}$ در صورت فرار در جهت $\lambda$، مسیر فرارش از سلول $c$ عبور کند. بنابر تعریف، اگر مکان اولیه‌ی $R_{i}$ روی سلول $c$ قرار گرفته‌باشد، آن‌گاه هر چهار زوج $(i, l)$، $(i, r)$، $(i, u)$ و $(i, d)$ در $P_{c}$ قرار دارند و در غیر این صورت، حد‌اکثر یکی از این چهار زوج، عضو $P_{c}$ است.

اکنون می‌توان محدودیت‌های خطی زیر را با این فرض که بیشینه چگالی نقاط برد برابر $Z$ باشد، نوشت:

\شروع{فقرات}

\فقره به ازای هر $i$، باید $x_{i, l} + x_{i, r} + x_{i, u} + x_{i, d} \geq 1$. به بیان دیگر، هر مستطیل باید حد‌اقل به یکی از چهار جهت فرار کند.

\فقره چگالی همه‌ی سلول‌ها باید کم‌تر یا مساوی $Z$ باشد، پس به ازای هر سلول $c$، محدودیت $\Sigma_{(i, \lambda) \in P_{c}} x_{i, \lambda} \leq Z$ وجود دارد.

\پایان{فقرات}

بنابر آن‌چه گفته شد، برنامه‌ریزی صحیح زیر معادل مسئله‌ی~\ref{prob:REP} است. دقت کنید که در این برنامه‌ریزی صحیح، تعداد محدودیت‌ها از $O(n ^ 2)$ است.
\begin{latin}
\begin{alignat}{3}
    & \text{minimize}  \quad && \ Z \notag \\
    & \text{subject to}  \quad 
		&& \!\!{\sum_{(i,\lambda) \in P_c} x_{i,\lambda} \leq Z} && \qquad {\forall c} \notag \\
    		&&& {x_{i,l} + x_{i,r} + x_{i,u} + x_{i,d} \geq 1} && \qquad {\forall\ 1 \le i \le n} \notag
\end{alignat}
\end{latin}
همان گونه که پیش‌تر اشاره شد، دامنه‌ی همه‌ی متغیر‌های جهت در برنامه‌ریزی صحیح داده‌شده، مجموعه‌ی $\set{0, 1}$ است. دامنه‌ی متغیر $Z$ را نیز می‌توان مجموعه‌ی همه‌ی اعداد طبیعی در نظر گرفت، هرچند می‌دانیم که مقدار $Z$ در جواب بهینه، عضو $\set{1, \ldots, n}$ خواهد بود. برنامه‌ریزی خطی متناظر با این برنامه‌ریزی صحیح را می‌توان این گونه تعریف کرد که همه‌ی محدودیت‌ها مطابق برنامه‌ریزی صحیح باشند، ولی دامنه‌ی متغیر‌های جهت در آن، همه‌ی اعداد گویای بازه‌ی $[0, 1]$ در نظر گرفته‌شود و دامنه‌ی $Z$ هم همه‌ی اعداد گویای بازه‌ی $[1, n]$.

هر‌چند با فرض $NP \neq P$، برای حل برنامه‌ریزی صحیح مورد نظر هیچ الگوریتم کارایی وجود ندارد، ولی برنامه‌ریزی خطی متناظر را می‌توان در زمان چند‌جمله‌ای حل کرد. پاسخ برنامه‌ریزی خطی را $OPT_{LP}$ می‌نامیم. فرض کنید که مقدار متغیر $x_{i, \lambda}$ در $OPT_{LP}$ را ${x^{*}}_{i, \lambda}$ و مقدار متغیر $Z$ در $OPT_{LP}$ را نیز $Z^{*}$ بنامیم. لازم به ذکر است که هر بردار امکان‌پذیر برای برنامه‌ریزی صحیح داده‌شده یک بردار امکان‌پذیر برای برنامه‌ریزی خطی متناظر نیز هست، پس برای برنامه‌ریزی صحیحی مورد نظر، بردار امکان‌پذیری نمی‌توان یافت که در آن مقدار متغیر $Z$ کم‌تر از $Z^{*}$ باشد. به بیان دیگر، کمینه چگالی ممکن برای فراری دادن مستطیل‌ها  نمی‌تواند کم‌تر از  $Z^{*}$ باشد.

اکنون با داشتن $OPT_{LP}$ می‌توان الگوریتم‌هایی تقریبی برای مسئله‌ی~\ref{prob:REP} به دست آورد. در این بخش ابتدا یک الگوریتم تقریبی ساده با ضریب تقریب $4$ ارائه خواهد شد. این الگوریتم پیش‌تر در \cite{REP} معرفی شده‌است. سپس الگوریتمی احتمالی ارائه خواهیم کرد که به ازای هر $\epsilon > 0$، در صورتی که جواب مسئله راه فرار مستطیل‌ها به اندازه‌ی کافی بزرگ باشد، با احتمال بسیار بالایی پاسخی با ضریب تقریب $1 + \epsilon$ به دست می‌دهد.

\قسمت{ضریب تقریب $4$}

با داشتن $OPT_{LP}$ و با بهره‌گیری از روش گرد کردن قطعی به سادگی می‌توان یک الگوریتم تقریبی با ضریب تقریب $4$ برای مسئله به دست آورد. برای این کار، با داشتن $OPT_{LP}$، یک بردار امکان‌پذیر برای برنامه‌ریزی صحیح اولیه، مطابق با آن‌چه در ادامه می‌آید، به دست می‌آوریم.

\شروع{فقرات}

\فقره برای هر مستطیل $i$، جهت $\lambda$ را به گونه‌ای انتخاب می‌کنیم که مقدار ${x^{*}}_{i, \lambda}$ در بین چهار متغیر‌ جهت مربوط به این مستطیل بیشینه باشد. اگر بیش از یک جهت با این ویژگی وجود داشت، یک جهت به دلخواه انتخاب می‌شود. سپس در برنامه‌ریزی صحیح معادل با مسئله‌ی~\ref{prob:REP}، مقدار متغیر $x_{i, \lambda}$ را برابر $1$ و مقدار سه متغیر جهت دیگری که مربوط به $R_{i}$ هستند را $0$ در نظر می‌گیریم. این مقدار‌دهی به متغیر‌ها معادل با این است که مستطیل $R_{i}$ در جهت $\lambda$ فرار کند.

\فقره مقدار متغیر $Z$ در برنامه‌ریزی صحیح مورد نظر را $\lfloor {4 {Z^{*}}} \rfloor$ قرار می‌دهیم.

\پایان{فقرات}

پیش از هر چیز نشان می‌دهیم بردار صحیحی که این گونه به دست می‌آید، یک بردار امکان‌پذیر است. برای این منظور دقت کنید که اگر مستطیل $R_{i}$ در جهت $\lambda$ فرار کند،  در $OPT_{LP}$ مقدار ${x^{*}}_{i, \lambda}$ حد‌اقل ${{1} \over {4}}$ بوده‌است، چراکه $x_{i, l} + x_{i, r} + x_{i, u} + x_{i, d} \geq 1$. بنابراین، در نتیجه‌ی گرد کردن، مقدار هر متغیر جهت، حد‌اکثر $4$ برابر شده‌است. پس اگر مقدار متغیر $Z$ در برنامه‌ریزی صحیح مورد نظر برابر $\lfloor {4 {Z^{*}}} \rfloor$ قرار داده‌شود، همه‌ی محدودیت‌های به شکل $\Sigma_{x \in P_{c}} x \leq Z$ رعایت خواهند شد. پس بردار به دست آمده یک بردار امکان‌پذیر برای برنامه‌ریزی صحیحی است که ارائه شده‌است.

از سوی دیگر، مقدار تابع هدف به ازای این بردار به دست آمده برابر همان $\lfloor {4 {Z^{*}}} \rfloor$ است و به این ترتیب می‌توان نتیجه گرفت که الگوریتم گفته‌شده، الگوریتمی تقریبی برای مسئله‌ی راه فرار مستطیل‌ها با ضریب تقریب $4$ است.

\قسمت{ضریب تقریب هر اندازه نزدیک به $1$}

اکنون به معرفی یک الگوریتم تقریبی با ضریب تقریبِ به میزان دلخواه نزدیک به $1$ در هنگامی که پاسخ بهینه به اندازه‌ی کافی بزرگ باشد، می‌پردازیم. به طور دقیق‌تر، الگوریتم ارائه‌شده در این بخش یک الگوریتم تصادفی است که اگر $Z^{*}$ (پاسخ بهینه‌ی برنامه‌ریزی خطی) از  $c_\epsilon \log n$، بیش‌تر باشد، آن گاه با احتمال زیاد جواب بدست آمده از $1 + \epsilon$ برابر جواب بهینه بیش‌تر نیست. دقت کتید که $c_{\epsilon}$ عدد ثابتی است وابسته به $\epsilon$ (و مستقل از $n$ یا هر پارامتر دیگری).

این الگوریتم، بر پایه‌ی گرد کردن تصادفی است که در بخش های پیشین به طور مختصر معرفی شده‌است. یک نکته‌ی مهم در این الگوریتم، مستقل نبودن متغیرهای تصادفی حاصل از گرد کردن متغیرهای برنامه‌ریزی خطی است. به بیان دیگر، در روشی که ارائه می‌کنیم، بر خلاف روشی که توضیح داده شده، متغیر‌ها به صورت مستقل از هم گرد نمی‌شوند، بلکه در گرد کردن آن‌ها وابستگی وجود دارد. ابتدا لم زیر را برای گرد کردن متغیرهای گویا به اعداد صحیح ارائه می‌دهیم.

\شروع{لم}

فرض کنید متغیرهای $x_i \in [0,1]$ داده شده‌اند و داریم $\Sigma x_i = 1$. می‌توان متغیرهای $x_i$ را به یکی از اعداد $0$ یا $1$ گرد کرد به گونه‌ای که:

\شروع{فقرات}

\فقره احتمال گرد شدن متغیر $x_i$ به یک، برابر با مقدار $x_i$ باشد.

\فقره یک و تنها یکی از متغیرهای داده شده $1$ شود و باقی همگی به $0$ گرد شوند.

\پایان{فقرات}

\پایان{لم}

\شروع{اثبات}

لازم به یادآوری است که روش گرد کردن متغیرها به صورت مستقل و با احتمال $x_i$ به ازای هر متغیر، در این جا کاربرد ندارد؛ چرا که در این روش، شرط دوم لم در برآورده نمی‌شود.

برای آن که بتوان هر دو شرط را به طور هم‌زمان برآورده کرد، روش مقابل را پیشنهاد می دهیم: به هرکدام از متغیرهای $x_i$ یک زیر‌بازه از بازه‌ی $[0, 1)$ با طولی برابر با مقدار $x_i$ نسبت می‌دهیم به طوری که هیچ دو بازه‌ای اشتراک نداشته باشند. به این ترتیب، از آن جایی که $\Sigma x_i = 1$، اجتماع این بازه‌های مجزا، کل $[0, 1)$ را می‌پوشاند.

پس از این کار، یک عدد تصادفی به صورت یکنواخت در بازه‌ی $[0, 1)$ انتخاب می‌کنیم و آن متغیر $x_i$ را که نقطه‌ی انتخاب شده در بازه‌ی نسبت داده‌شده به آن باشد، به $1$ گرد می‌کنیم. سایر متغیرها را نیز به $0$ گرد می‌نماییم. با توجه به این که اندازه‌ی بازه‌ی نسبت داده‌شده به هر متغیر $x_i$ برابر است با مقدار آن متغیر، احتمال آن که یک متغیر به $1$ گرد شود، مساوی با مقدار $x_i$ است. از طرف دیگر، با توجه به این که اجتماع این بازه‌های مجزا تمام بازه‌ی $[0, 1)$ را می‌پوشانند و نقطه‌ی انتخاب شده در یک و تنها یک بازه خواهد بود، شرط دوم لم با این روش برآورده می‌شود.

\پایان{اثبات}

اکنون با استفاده از لم بالا، الگوریتم تقریبی زیر را برای مسئله‌ی~\ref{prob:REP} ارائه می‌دهیم.
  
\شروع{فقرات}

\فقره برای هر مستطیل $i$ داریم $x^{*}_{i, l} + x^{*}_{i, r} + x^{*}_{i, u} + x^{*}_{i, d} = 1$. مطابق لم بالا، این چهار متغیر جهت را گرد می کنیم. فرض کنید که متغیر ${\hat{x}_{i, \lambda}}$ را برابر با مقدار گرد شده متغیر ${x^{*}}_{i, \lambda}$ در نظر بگیریم. به این ترتیب، از بین چهار متغیر ${\hat{x}}_{i, l}$، ${\hat{x}}_{i, r}$، ${\hat{x}}_{i, u}$ و ${\hat{x}}_{i, d}$، مقدار یکی برابر $1$ و مقدار سه‌تای دیگر $0$ است.

\فقره مستطیل $R_{i}$ را به جهتی فرار می‌دهیم که مقدار ${\hat{x}}_{i, \lambda}$ متناظر با آن جهت برابر $1$ باشد.

\پایان{فقرات}

\شروع{قضیه}

الگوریتم گفته‌شده یک الگوریتم تقریبی با ضریب تقریب $1+\epsilon$ برای مسئله‌ی راه فرار مستطیل‌ها است، هنگامی که $Z^{*} \geq (9/\epsilon^{2}) \ln n$.

\پایان{قضیه}

\شروع{اثبات}

همان گونه که توضیح داده‌شد، این الگوریتم به هر مستطیل یک و تنها یک جهت برای فرار نسبت می‌دهد. حال چگالی نقطه‌های قاب را با جواب $Z^{*}$ مقایسه می‌کنیم. به ازای هر سلول در قاب مانند $c$، متغیر $d_c$ را برابر با چگالی آن سلول وقتی مستطیل‌ها طبق الگوریتم گفته‌شده فرار می‌کنند، تعریف می‌کنیم. بنابر این داریم:
$$d_c = \Sigma_{(i, \lambda) \in P_c} {{\hat{x}}_{i, \lambda}}$$

طبق روش گرد کردن می‌توان گفت:
$$E[d_c] = E[\Sigma_{(i, \lambda) \in P_c} {\hat{x}}_{i, \lambda}] =  \Sigma_{(i, \lambda) \in P_c} E[{\hat{x}}_{i, \lambda}] =  \Sigma_{(i, \lambda) \in P_c} P[{\hat{x}}_{i, \lambda}] =  \Sigma_{(i, \lambda) \in P_c} {x^{*}}_{i, \lambda} \leq Z^{*}$$

در عبارت بالا، تساوی دوم بر اساس خطی بودن امید ریاضی به دست آمده، تساوی سوم بر اساس قسمت شرط نخست بالا و تساوی چهارم بر اساس تعریف امید ریاضی برای متغیر‌های تصادفی شناسه\زیرنویس{به فصل مقدمات رجوع کنید}. نامساوی پایانی نیز نتیجه‌ی محدودیت‌های برنامه‌ریزی خطی است.  بنابر آن چه به دست آمد، امید ریاضی چگالی هر سلول کم‌تر از یا مساوی با  $Z^{*}$  است. اکنون باید فاصله‌ی مقدار چگالی یک سلول از مقدار امید ریاضی آن را بررسی کنیم. ابزاری که برای این کار استفاده می‌کنیم، نامساوی چرنوف است که پیش از این مطرح شده‌است.

مطابق با آن‌چه در نامساوی چرنوف گفته شده، در این جا، متغیر تصادفی $d_c$ مجموع تعدادی متغیر تصادفی است که دامنه‌ی آنها $\set{0, 1}$ است. تنها نکته‌ای که باید در نظر داشت این است که در نامساوی چرنوف، شرط مستقل بودن متغیرها وجود دارد، در حالی که روش گرد کردن متغیرها در الگوریتم بالا مستقل نیست. در این باره می‌توان گفت:

\شروع{فقرات}

\فقره اگر مکان اولیه‌ی مستطیل $R_{i}$ روی سلول $c$ قرار داشته‌باشد، آن گاه هر چهار زوج $(i, l)$، $(i, r)$، $(i, u)$ و $(i, d)$ عضو $P_c$ هستند. در این حالت، به جای چهار متغیر وابسته به $R_{i}$ در عبارت $d_c = \Sigma_{(i, \lambda) \in P_c} {{\hat{x}}_{i, \lambda}}$، می‌توان عدد $1$ قرار داد؛ چرا که همواره یک و تنها یکی از این متغیر‌ها مقدار $1$ خواهد داشت.

\فقره اگر مکان اولیه‌ی مستطیل $R_{i}$ روی سلول $c$ قرار نداشته‌باشد، آن گاه حد‌اکثر یکی از چهار زوج $(i, l)$، $(i, r)$، $(i, u)$ و $(i, d)$  در $P_c$ است. از سوی دیگر، به سادگی می‌توان دید که متغیر‌های ${\hat{x}}_{i, \lambda}$ و ${\hat{x}}_{i', \lambda'}$ به ازای $i \neq i'$ مستقل از هم هستند.

\پایان{فقرات}

بنابر آن چه گفته شد، برای هر سلول $c$، می‌توان فاصله‌ی $d_c$ از امید ریاضی آن را با کمک نامساوی چرنوف بررسی کرد.
$$P(d_c \geq (1+\epsilon) E[d_c]) \leq (\frac{e^{\epsilon}}{(1+\epsilon)^{(1+\epsilon)}})^{Z^{*}}$$ 
هم چنین، با توجه به آن چه که پیش‌تر در رابطه با نامساوی چرنوف گفته شد، می‌توان نوشت:
$$(\frac{e^{\epsilon}}{(1+\epsilon)^{(1+\epsilon)}})^{Z^{*}} \leq e^{-{Z^{*}}\epsilon^2/3}$$

دقت کنید پاسخی که الگوریتم ما برای مسئله‌ی راه فرار مستطیل‌ها تولید می‌کند، برابر است با بیشینه مقدار $d_c$ یا به بیان دقیق‌تر $max_c\set{d_c}$. هم‌چنین تعداد سلول‌های قاب حد‌اکثر برابر است با $(2n)^{2}$. حال فرض کنید که به ازای یک ثابت عددی وابسته به $\epsilon$ مانند $c_{\epsilon}$ داریم:
$$Z^{*} \geq c_{\epsilon}  \ln n$$ 
در نتیجه خواهیم داشت: 
$$P\{max_c \set{d_c} \geq (1+\epsilon){Z^{*}} \} \leq \Sigma_{c} P\{d_c \geq (1+\epsilon) {Z^{*}}\} \leq (2n)^{2} \times n^{-c_{\epsilon}\epsilon^{2}/3}$$

بنا بر آن چه گفته شد، اگر ثابت عددی $c_{\epsilon}$ برابر با $9/{\epsilon}^2$ در نظر گرفته‌شود، احتمال آن که بیشینه چگالی در الگوریتم ما از $1+\epsilon$ برابر پاسخ برنامه‌ریزی خطی بیش‌تر باشد، از  $\frac{4}{n}$کم‌تر است. از آن جایی که $Z^{*}$ کران پایینی برای مسئله‌ی~\ref{prob:REP} است، می‌توان ادعا کرد وقتی $Z^{*} \geq c_{\epsilon}  \ln n$، آن گاه با احتمال بالا خروجی الگوریتم ما حداکثر $1+\epsilon$ برابر پاسخ بهینه است.
 
\پایان{اثبات}

