
\فصل{نتیجه‌گیری}

در این رساله، نتایج جدیدی در مورد مسئله‌ی راه فرار مستطیل‌ها به دست آمد که به طور خلاصه به شرح زیر هستند:

\شروع{فقرات}

\فقره ثابت شد \decision{2} در کلاس پیچیدگی \کامل{} قرار دارد. این در حالی است که پیش از این، \کامل{} بودن این مسئله به ازای $k \geq 3$ در \cite{REP} نشان داده شده‌بود و از سوی دیگر برای $k = 1$، \cite{BoundaryRec} الگوریتمی چند‌جمله‌ای ارائه کرده‌بود. به این ترتیب، وضعیت مسئله‌ی~\ref{prob:k-REP} برای هر عدد $k$ مشخص شد.

\فقره با فرض $NP \neq P$، نمی‌توان الگوریتمی تقریبی با زمان چند‌جمله‌ای و ضریب تقریب بهتر از ${3} \over {2}$ برای مسئله‌ی~\ref{prob:REP} یافت.

\فقره پاسخ \decision{1} را می‌توان در زمان $O(n ^ 4)$ یافت که در مقایسه با الگوریتم ارائه شده در \cite{BoundaryRec} از زمان اجرای بهتری برخوردار است و در عمل می‌تواند بسیار قابل استفاده‌تر باشد.

\فقره به ازای هر عدد $\epsilon > 0$، برای مسئله‌ی~\ref{prob:REP} الگوریتمی تقریبی و احتمالی با ضریب تقریب $1 + \epsilon$ ارایه شد، ولی به شرط آن که پاسخ از عدد مشخصی (وابسته به $\epsilon$) بیش‌تر باشد.

\پایان{فقرات}

علی‌رغم الگوریتم تقریبی ارائه شده در این مقاله، هنوز یک پرسش جذاب در رابطه با این مسئله بی‌پاسخ مانده‌است: آیا می‌توان الگوریتمی تقریبی با ضریب تقریب بهتر از $4$ برای مسئله‌ی راه فرار مستطیل‌ها در حالت کلی یافت؟
