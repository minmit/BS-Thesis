
\فصل{پروتکل پیشنهادی}

در این فصل به ارائه‌ی پروتکل پیشنهادی خود برای مسیریابی در شبکه‌های مبتنی بر داده‌های نام‌گذاری شده می‌پردازیم. لازم به ذکر است همان‌طور که پیش‌تر در توضیح دامنه‌ی ارسال مسیریاب‌های NDN به آن اشاره شد، به دلیل انعطاف‌پذیر بودن و پشتیبانی از چندمسیری در این دامنه‌، بسیاری از مشکلات پیش آمده در تحویل بسته‌ها در همین دامنه مدیریت می‌شود و در نتیجه بخشی از انتظارات از قسمت مسیریابی برداشته می‌شود. به عنوان مثال اطمینان از ایجاد نشدن حلقه در مسیرها از مهم‌ترین ویژگی‌های یک پروتکل مسیریابی در شبکه‌های مبتنی بر IP می‌باشد،در حالی که به دلیل معماری NDN در دامنه‌ی ارسال از ایجاد حلقه جلوگیری می‌شود و این مسئله دیگر در نیازمندی‌های دامنه‌ی مسیریابی نیست. 

 با توجه به ساختار شبکه‌های NDN، هر پروتکلی که بخواهد روی این شبکه اجرا شود باید به چهار سوال مهم در مورد موضوعات زیر پاسخ دهد:
\شروع{فقرات}
\فقره \مهم{نام‌گذاری}: چگونگی نام‌گذاری مسیریاب‌ها، پیوند‌ها، و پیغام‌های مسیریابی
\فقره \مهم{مدل اعتماد\زیرنوشت{Trust Model}}: چگونگی توزیع کلید‌های رمز مسیریاب‌ها و اطمینان از صحت آن‌ها
\فقره \مهم{انتشار اطلاعات}: چگونگی انتشار پیغام‌های به‌روزرسانی مسیریابی در سرتاسر شبکه
\فقره \مهم{چندمسیری}: چگونگی پیدا کردن مسیرهای مختلف به یک پیشوند و رتبه‌بندی آن‌ها برای پشتبانی از چندمسیری
\پایان{فقرات}

در ادامه، از میان چهار موضوع بالا، در مورد نام‌گذاری، نحوه‌ی انتشار اطلاعات، و پشتیبانی از چندمسیری در پروتکل پیشنهادی خود صحبت خواهیم کرد. طراحی مدل اعتماد به دلیل نیاز به  پیش‌نیازهایی فراوان به آینده موکول خوهد شد. 

\قسمت{نام‌گذاری}

یکی از بخش‌های مهم در پروتکل مسیریابی، تعیین نام‌گذاری مسیریاب‌ها و پیغام‌هاست. از آن جایی که مبنای شبکه‌های NDN داده‌های نام‌گذاری شده است، اگر یک پروتکل مسیریابی بخواهد به طور مستقیم روی این شبکه‌ها اجرا شود نیازمند مکانیزم نام‌گذاری متناسب با آن‌هاست. در این بخش به توضیح نام‌گذاری مسیریاب‌ها و پیشوند پیغام‌ها می‌پردازیم و نحوه‌ی نام‌گذاری دقیق‌تر پیغام‌ها را در بخش بعدی بررسی خواهیم کرد. 

با توجه به ساختار سلسله‌مراتبی شبکه، به نظر می‌رسد نام‌گذاری سلسله‌مراتبی مناسب‌ترین گزینه برای بیان ارتباط بین مولفه‌های مختلف آن باشد. بدین منظور برای هر شبکه و هر مسیریاب یک نام در نظر می‌گیریم و کافی است که نام شبکه‌ها و نیز نام مسیریاب‌ها در هر شبکه یکتا باشند. در آن صورت می‌توانیم یک مسیریاب را با نام سلسله‌مراتبی \کج{ /<network>/<router>} بشناسیم که در آن \کج{<network>} نام شبکه و \کج{<router>} نام مسیریاب است. به عنوان مثال نام یک مسیریاب در دانشگاه شریف می‌تواند \کج{/sharif/ce-router} باشد. 

در مورد نام‌گذاری پیغام‌ها باید به این نکته توجه کرد که استفاده از مخزن CCNx مستلزم این است که نام داده‌های یک مجموعه در پیشوندی که نام مجموعه است مشترک باشند. به همین دلیل برای تمام بسته‌های مربوط به مسیریابی از پیشوند \کج{/<network>/RM} استفاده می‌کنیم که در آن \کج{<network>} نام شبکه است و \کج{RM} مخفف Routing Message و برای متمایز کردن بسته‌های مسیریابی از سایر بسته‌هاست. در ادامه‌ی نام هر بسته نوع آن، نام مسیریاب تولید کننده‌ی آن، و عدد نسخه‌ی آن به شکل \کج{/<router>/<type>/<version>} قرار می‌گیرند. در بخش‌های بعدی در مورد نوع بسته‌ها و نسخه‌بندی آن‌ها بیشتر توضیح خواهیم داد.

\قسمت{پیغام‌ها}
پیغام‌ها




