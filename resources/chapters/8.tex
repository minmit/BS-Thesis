\فصل{نتیجه‌گیری}

در این رساله پروتکلی مبتنی بر بردار فاصله برای مسیریابی در شبکه‌های مبتنی بر داده‌های نام‌گذاری‌شده پیشنهاد داده‌ایم. در این پروتکل دو نوع بسته‌ برای تبادل اطلاعات مسیریابی تعریف کردیم. بسته‌های اعلام‌کننده‌ی فاصله‌، اطلاعات مربوط به مسیرهای بین مسیریاب‌ها را جابه‌جا می‌کنند، و بسته‌های اعلام‌کننده‌ی پیشوند، پیشوند‌های تولید‌شده در شبکه را توزیع می‌کنند. این پروتکل روی NDN اجرا می‌شود به این معنی که از بسته‌های درخواست و داده‌ی خود NDN استفاده می‌کند. نکاتی که این پروتکل را از سایر پروتکل‌های مسیریابی مطرح‌شده برای NDN متمایز می‌کند به شرح زیرند:
\شروع{فقرات}
\فقره از آن جایی که این پروتکل مبتنی بر بردار فاصله است، تعداد پیغام‌های حاوی اطلاعات مسیرهای بین مسیریاب‌ها، نسبت به پروتکل‌های دیگر که مبتنی بر وضعیت پیوند هستند، به مراتب کمتر است. 
\فقره الگوریتم پیدا کردن مسیرهای مختلف به یک داده‌ی یکسان، نسبت به الگوریتم‌های مشابه در پروتکل‌های دیگر سربار محاسباتی کمتری دارد.
\فقره برای توزیع بسته‌ها در شبکه، از مخازن و پروتکل Sync در CCNx استفاده می‌شود. هم‌چنین پروتکل Sync ، برای پشتیبانی بهتر از حذف بسته‌ها تغییر داده شده است تا مناسب استفاده در پروتکل مسیریابی پیشنهادی شود.
\پایان{فقرات}

در کنار ویژگی‌های مثبت بالا، پروتکل پیشنهادی با استفاده از روش سم معکوس سعی در کاهش تاثیر مسئله‌ی شمارش بی‌انتها دارد که در پروتکل‌های مبتنی بر بردار فاصله مشکل مهمی به حساب می‌آید. علاوه بر آن، پشتیبانی از چندمسیری در دامنه‌ی ارسال در شبکه‌های NDN اثر مشکل شمارش بی‌انتها را کمتر می‌کند. به همین دلیل این پروتکل، به ویژه برای شبکه‌های محلی و کوچک‌تر، انتخاب مناسبی به نظر می‌رسد.

با وجود بهبودهایی که این پروتکل در زمینه‌های مختلف ایجاد کرده است، فضا برای بهبود بیشتر آن وجود دارد. از کارهایی که در آینده می‌تواند به استفاده از این پروتکل در عمل کمک کند، تعریف یک مدل اعتماد برای مدیریت و توزیع کلید‌های رمزنگاری مسیریاب‌هاست. هم‌چنین پیدا کردن روشی برای حل مسئله‌ی شمارش بی‌انتها، یا کاهش بیشتر تاثیر آن از اهدافی است که در ادامه دنبال خواهیم کرد.

