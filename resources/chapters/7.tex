\فصل{ارزیابی}

در این فصل به ارزیابی پروتکل پیشنهادی خود می‌پردازیم. همان‌طور که در فصل‌های قبل و به ویژه ~\ref{prevWorks} به آن اشاره شد، پروتکل NLSR قابل‌توجه‌ترین و جدیدترین پروتکل پیشنهادی مسیریابی در NDN است و به همین دلیل برای این ارزیابی، به مقایسه‌ی ویژگی‌های این پروتکل و پروتکل پیشنهادی خود می‌پردازیم.

یکی از ویژگی‌های مهم NDN پشتیبانی آن از چندمسیری در دامنه‌ی ارسال است و پروتکل‌های مسیریابی در این شبکه باید بتوانند تا حد ممکن از این قابلیت استفاده کنند. NLSR با چند بار اجرای الگوریتم کوتاه‌ترین مسیر روی گراف شبکه، برای هر پیشوند واسط‌های مسیریاب را به ترتیب فاصله‌ی آن‌ها از تولید‌کننده‌ی پیشوند رتبه‌بندی می‌کند. پروتکل پیشنهادی به دلیل مبتنی بودن بر بردار فاصله، می‌تواند همین کار را تنها با اجرای یک الگوریتم مرتب‌سازی روی اطلاعات کسب‌شده از بسته‌های DA از واسط‌های مختلف خود انجام دهد. بدین‌ترتیب سربار محاسبه‌ی مسیریاب کاهش چشمگیری خواهد داشت.

در مورد مسیریابی پیشوند‌ها، پروتکل NLSR و پروتکل پیشنهادی ما، عملکرد مشابهی دارند. در هر دو پروتکل، بسته‌های اعلام‌کننده‌ی پیشرفت با استفاده از پروتکل Sync در سرتاسر شبکه توزیع می‌شوند، زیرا برای مسریابی درست، تمام مسیریاب‌ها باید از تولیدکنندگان مختلف یک داده اطلاع داشته باشند. این امر به دلیل زیاد بودن داده‌ها برای شبکه سربار قابل‌توجهی تولید خواهد کرد ولی بدون چشم‌پوشی از بخشی از داده‌ها، از آن گریزی نیست. یکی از راه‌حل‌هایی که برای این مسئله به نظر می‌رسد استفاده از روشی مشابه \cite{two-layer} است که در آن داده‌هایی که درخواست کمتری دارند به کل شبکه توزیع نمی‌شود. یافتن راهی مناسب برای کم‌کردن این سربار از کارهای آینده‌ی ما خواهد بود.

تفاوت اصلی بین NLSR و پروتکل پیشنهادی ما در نحوه‌ی توزیع بسته‌های مربوط به مسیریابی بین مسیریاب‌هاست. از آن‌جایی که NLSR پروتکلی مبتنی بر وضعیت پیوند است، بسته‌های مربوط به این نوع مسیریابی نیز مشابه بسته‌های اعلام‌کننده‌ی پیشوندها در سرتاسر شبکه توزیع می‌شوند. این در حالی است که در پروتکل پیشنهادی ما که مبتنی بر بردار فاصله است، ارسال بسته‌های DA تنها در صورتی انجام می‌گیرد که بهترین مسیر در 
