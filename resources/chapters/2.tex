
\فصل{معرفی مسئله}

\قسمت{تعریف دقیق مسئله}

در این فصل، مسئله‌ی راه فرار مستطیل‌ها\زیرنوشت{Rectange Escape Problem (REP)} را معرفی خواهیم کرد. این مسئله را می‌توان این گونه تعریف کرد:

\شروع{مسئله}[راه فرار مستطیل‌ها]
\label{prob:REP}

یک ناحیه‌ی مستطیلی با مرز‌های موازی محور‌های مختصات به نام \کج{قاب} داده شده‌است که درون آن $n$ مستطیل با اضلاع موازی محور‌ها قرار گرفته‌اند. هدف، فراری دادن این مستطیل‌ها - هر یک به یکی از چهار جهت اصلی - است، به گونه‌ای که بیشینه چگالی\زیرنوشت{Density} نقاط قاب کمینه شود. چگالی یک نقطه از قاب، تعداد مستطیل‌هایی تعریف می‌شود که روی آن نقطه قرار گرفته‌اند یا در فرارشان از آن نقطه عبور می‌کنند.

\پایان{مسئله}

\شکل‌پی‌دی‌اف{8}{یک نمونه از مسئله‌ی راه فرار مستطیل‌ها}{example}

در این رساله، برای سادگی، مسئله‌ی تصمیم‌گیری زیر را نیز تعریف می‌کنیم.

\شروع{مسئله}
\label{prob:k-REP}

یک نمونه از مسئله‌ی راه فرار مستطیل‌ها و عدد طبیعی $k$ داده شده‌اند. آیا می‌توان مستطیل‌ها را به گونه‌ای فراری داد که چگالی هیچ نقطه‌ای از قاب بیش‌تر از $k$ نشود؟

\پایان{مسئله}

بدیهی است که به ازای یک ورودی از مسئله‌ی راه فرار مستطیل‌ها، پاسخ مسئله‌ی~\ref{prob:REP} برابر است با کم‌ترین عدد $k$ که پاسخ مسئله‌ی~\ref{prob:k-REP} به ازای آن، \بلی{} شود.

مسئله‌ی راه فرار مستطیل‌ها که در \cite{REP} تعریف شده‌است، در مسیر‌دهی روی برد‌های دیجیتال\زیرنوشت{Printed Circuit Board (PCB) Routing} مورد نیاز است. تراشه‌های مستطیلی شکلی را در نظر بگیرید که روی یک برد قرار گرفته‌اند، به گونه‌ای که اضلاع تراشه‌ها موازی اضلاع برد است. می‌خواهیم هر یک از تراشه‌ها را در یکی از چهار جهت اصلی از طریق یک گذر‌گاه\زیرنوشت{Bus} (مطابق شکل~\ref{fig:example}) به دیواره‌ی برد متصل کنیم. هدف، کمینه کردن بیش‌ترین تعداد گذرگاهی است که در یک نقطه برخورد می‌کنند تا تعداد لایه‌های لازم برای اتصال تراشه‌ها به دیواره‌های قاب کمینه شود.

مسئله‌ی کمینه‌سازی تعداد لایه‌ها، پیش‌تر نیز مورد بررسی قرار گرفته‌اند~\cite{BoundaryRec,KongBUSPLANNER,MaREP,MaOPT,motiveOZDAL,WuILP,Yan12,Yan2012,KongBIPARTITE}. گفتنی است که ~\cite{BoundaryRec} برای مسئله‌ی زیر الگوریتمی با زمان اجرای $O(n ^ 6)$ ارائه کرده‌است:

\شروع{مسئله}
\label{prob:max-disjoint-rec}

بر روی یک قاب مستطیلی، $n$ مستطیل با اضلاع موازی اضلاع قاب داده شده‌اند، به گونه‌ای که از هر مستطیل، حد‌اقل یک ضلع روی مرز قاب قرار گرفته‌است. بیش‌ترین تعداد مستطیل از میان این $n$ مستطیل را بیابید که هم‌پوشانی نداشته باشند.

\پایان{مسئله}

به سادگی می‌توان دید که با کمک مسئله‌ی~\ref{prob:max-disjoint-rec} می‌توان \decision{1} را حل کرد. کافی است برای یک نمونه از مسئله‌ی راه فرار مستطیل‌ها، همه‌ی $n$ مستطیل داده‌شده را به هر چهار جهت تا مرز قاب گسترش دهیم. به این ترتیب، $4n$ مستطیل خواهیم داشت که حد‌اقل یک ضلع از هر کدام از آن‌ها روی مرز قاب قرار گرفته‌است. اکنون کافی است از میان این $4n$ مستطیل، بیش‌ترین تعداد مستطیل بدون هم‌پوشانی را بیابیم. این تعداد برابر $n$ است اگر و تنها اگر مستطیل‌ها بتوانند با بیشینه چگالی $1$ فرار کنند.

\قسمت{کار‌های پیشین}

درباره‌ی مسئله‌ی راه فرار مستطیل‌ها پیش از این ثابت شده‌است:

\شروع{فقرات}

\فقره مسئله‌ی راه فرار مستطیل‌ها یک مسئله‌ی \سخت{} است. به طور دقیق‌تر، مسئله‌ی~\ref{prob:k-REP} به ازای $k = 3$ در کلاس پیچیدگی \کامل{} قرار دارد~\cite{REP}.

\فقره هر‌چند مسئله‌ی راه فرار مستطیل‌ها در حالت کلی \سخت{} است، ولی همان گونه که توضیح داده‌شد، برای مسئله‌ی~\ref{prob:k-REP} به ازای $k = 1$، یک الگوریتم با زمان اجرای $O(n ^ 6)$ در \cite{BoundaryRec} ارائه شده‌است.

\فقره در \cite{REP}، الگوریتمی تقریبی با زمان اجرای چند‌جمله‌ای و ضریب تقریب $4$ برای مسئله‌ی \ref{prob:REP} ارائه شده‌است.

\پایان{فقرات}

\صفحه‌جدید

\قسمت{نتایج ما}

نتایجی که در این رساله به دست آورده‌ایم، به شرح زیر است:

\شروع{فقرات}

\فقره مسئله‌ی~\ref{prob:k-REP} به ازای $k = 2$ در کلاس پیچیدگی \کامل{} قرار دارد. گفتنی است که این نتیجه، حتی برای حالت خاص‌تر مسئله که در آن هیچ دو مستطیلی از $n$ مستطیل ورودی هم‌پوشانی ندارند هم برقرار است.

\فقره با فرض $P \neq NP$، برای مسئله‌ی~\ref{prob:REP} نمی‌توان الگوریتمی تقریبی با ضریب تقریب بهتر از ${3} \over {2}$ یافت.

\فقره مسئله‌ی~\ref{prob:k-REP} وقتی $k = 1$، در زمان $O(n ^ 4)$ قابل حل است.

\فقره به ازای هر عدد ثابت $\epsilon > 0$، در حالتی که جواب مسئله‌ی~\ref{prob:REP} به اندازه‌ی کافی بزرگ باشد، الگوریتمی احتمالی با ضریب تقریب $1 + \epsilon$ برای مسئله‌ی~\ref{prob:REP} وجود دارد.

\پایان{فقرات}
