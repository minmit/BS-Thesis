
\فصل{مقدمه}

نقش و تاثیر گسترده‌ی اینترنت در دنیای امروزی بر هیچ‌کس پوشیده نیست. معماری اینترنت امروزی بر پایه لایه جهانی IP است. یعنی از IP به عنوان لایه شبکه در مدل OSI \زیرنوشت{Open Systems Interconnection} استفاده می‌شود. ماهیت این معماری بر اساس مکالمه بین دو انتهای ارتباط است. در این نوع معماری، هر شخص یک آدرس منحصر به فرد دارد و در حقیقت ارتباط بین آدرس‌ها برقرار می‌شود. 
در سال‌های اخیر، نیاز به توزیع داده‌ از طریق اینترنت به شدت افزایش پیدا کرده است. نه تنها داده‌هایی از نوع متن، بلکه نیاز به توزیع داده‌هایی از جنس عکس، صوت و فیلم هم رو بسیار زیاد شده است. می‌توان گفت که بسیاری از برنامه‌های امروزی نیاز به خود داده دارند و برای آن‌ها مهم نیست که این داده از طرف چه کسی به آن‌ها می‌رسد. همین‌طور با افزایش دستگاه‌های متحرک از قبیل تلفن‌های همراه، مشکلاتی مانند تغییر IP آن‌ها رو به افزایش است. این مشکلات باعث می‌شود که شبکه‌‌های مبتنی بر IP دیگر کارایی قبلی را در مقابل نیازهای امروزی نداشته باشند. شبکه‌های مبتنی بر داده‌‌های نام‌گذاری شده به تازگی برای رفع مشکلات شبکه‌های امروزی مطرح شده‌اند
\cite{ndn}.

در این شبکه‌ها این داده‌ها هستند که نام‌های منحصر به فرد دارند و دیگر آدرس مبدا و مقصدی وجود ندارد. در حقیقت داده‌ها و نام‌‌های آن‌ها نقش اساسی را در شبکه بازی می‌کنند. در این شبکه‌ها، دو نوع بسته درخواست و داده داریم. مشتری با فرستادن یک بسته درخواست کار را آغاز می‌کند. این بسته شامل نام داده‌ای است که مشتری نیاز دارد. وقتی که این بسته به مسیریاب می‌رسد، مسیریاب ابتدا در بانک داده خود که جایی است که داده‌ها را ذخیره می‌کند به دنبال این داده می‌گردد. اگر این داده را پیدا کرد که جواب را برمی‌گرداند. در غیر این صورت از روی جدول اطلاعات ارسال (FIB) خود، واسطی که این داده را باید بر روی آن بفرستد را استخراج می‌کند و بسته را بر روی آن واسط می‌فرستد. بعد از آن، در جدول درخواست‌های معلق (PIT)، نام این داده را به همراه واسطی که این درخواست از طریق آن آمده بود را ثبت می‌کند. بسته به همین‌ طریق به مسیر خود ادامه می‌دهد تا اینکه به منبع خود برسد. تولیدکننده این داده، یک بسته داده بازمی‌گرداند. این بسته شامل نام داده موردنظر به همراه محتوایات داده است. وقتی که یک بسته داده به یک مسیریاب می‌رسد، مسیریاب نام این داده را در جدول PIT خود جستجو می‌کند و این بسته داده را به تمام واسط‌هایی که این داده را درخواست کرده بودند، می‌فرستد. این روند هم ادامه پیدا می‌کند تا بسته به دست مشتری نهایی برسد. 
 
